\documentclass[a4paper, 11pt]{article}
\usepackage[utf8]{inputenc}
\usepackage[T1]{fontenc}
\usepackage[slovene]{babel}
\usepackage{lmodern}

\usepackage{amsmath}
\usepackage{amsthm}
\usepackage{amsfonts}

\usepackage{booktabs}

\title{Peanovi aksiomi}
\author{Mojca Novak}
\date{27. avgust 2024}

{\newtheorem{definicija}{Definicija}
\theoremstyle{definition}}

\newcommand{\N}{\mathbb{N}}

% Peanovi aksiomi
% Mojca Novak

\begin{document}
\maketitle

\begin{center}
    \textbf{Povzetek}\\
    % povzetek
    V nadaljevanju je zapisana definicija Peanovih aksiomov.
\end{center}

        % definicija
\begin{definicija}\emph{
	        \textbf{Peanovi aksiomi} \cite{zapiski}
        Množica naravnih števil je množica $\N$ s funkcijo $\varphi:\N \rightarrow \N$, 
        ki vsakemu naravnemu številu $n$ priredi njegovega neposrednega naslednika $\varphi(n)$. 
        Pri tem veljajo naslednji aksiomi:
        \begin{enumerate}
            \item $\N$ vsebuje število $\epsilon$, ki ni neposredni naslednik nobenega naravnega števila;
            \item neposredna naslednika dveh različnih naravnih števil sta različna, tj.\ funkcija $\varphi$ je injektivna: 
            če je $n\ne m$, je $\varphi(n)\ne \varphi(m)$;
            \item Če za podmnožico $A\in \N$ veljata lastnosti:
            \begin{enumerate}
                \item $\epsilon \in A$ in
                \item če je $n \in A$, je tudi $\varphi(n) \in A$,
                potem je $A = \N$.
            \end{enumerate}
        \end{enumerate}}
\end{definicija}
        % konec definicije
        \bibliography{viri.bib}
        \bibliographystyle{plain}
        % literatura
    
\end{document}